\documentclass[twocolumn]{article}
\usepackage{fullpage}
\usepackage{hyperref}

\title{CS250: An Elliptic Curve Cryptography Engine}
\author{Palmer Dabbelt and Kevin Linger}
\date{October 8, 2013}

\begin{document}
\maketitle

For our CS250 project we will be implementing an elliptic curve
cryptography (ECC) engine that attaches via the Rocket accelerator
interface.  Specifically we will be implementing an accelerator for
the elliptic curve digital signature algorithm (ECDSA), a NIST
standard widely used on the Internet for digital signatures.

\section{Elliptic Curve Cryptography}

Digital signature algorithms\cite{fips-186-3} are a widely used class
of algorithms that are central to the functioning of the modern
Internet.  DSA\cite{us-dsa} (the original digital signature algorithm
as standardized by NIST) is a specialization of the ElGamal signature
scheme\cite{elgamal-sig} that uses discrete logarithms over finite
integer fields.  Finite integer fields are particularly easy to
compute with electronics, so this gives DSA the advantage of being an
efficient signature algorithm.

The problem is that the discrete logarithm problem is not particularly
strong over finite integer fields (specifically, there are known
sub-exponential algorithms\cite{adleman-subexp}).  As shown in
Table~\ref{key-sizes}, DSA does not scale well to higher security
levels.  This is important because code breaking algorithms are
getting more efficient, leading to the need for higher security
levels, therefore imposing higher computational costs on every node in
the system.

\begin{table}[h]
  \begin{center}
    \begin{tabular}{cccc}
      Security Level & DSA & ECDSA & Ratio \\
      \hline
      80 & 1024 & 160 & 3:1 \\
      112 & 2048 & 224 & 6:1 \\
      128 & 3072 & 256 & 10:1 \\
      192 & 7680 & 384 & 32:1 \\
      256 & 15360 & 521 & 64:1 \\
    \end{tabular}
  \end{center}

  \caption{Comparison of DSA and ECDSA\cite{nsa-case_for_ecc}
    \label{key-sizes}}
\end{table}

The commonly accepted solution to this problem is to change the field
from integers to elliptic curves, which (as shown in
Table~\ref{key-sizes}) scale much better to higher security levels.
While elliptic curves are significantly more difficult to build
electronics for, this will eventually be offset by sheer key size
alone.  The currently accepted security level is 128 bits, at which
ECDSA already has a 10:1 advantage in key size.

\section{ECDSA in Hardware}

Unlike finite integer fields, elliptic curves do not map directly to
the hardware present in current microprocessors\cite{kss-ecdsa}.  This
suggests that a hardware accelerator designed to compute over elliptic
curves should lead to a significantly more efficient implementation
than in software alone.

Previous projects\cite{nnll-ecdsa_hw} have shown that ECC hardware can
be implemented efficiently, but were limited in the amount of
design-space exploration that was attempted\cite{mmm-hw_ecc}.  There
are three interesting axes on which to do design space exploration
upon: the size of functional units, the types of functional units, and
the location of the control logic.

Despite the fact that ECC involves smaller keys, the numbers are still
quite large: 80 bits is the smallest viable key size, while 256 bits
is a reasonable but large key size.  As such, there should be an
interesting tradeoffs between having larger multipliers as opposed to
smaller, chained multipliers.

ECC demands not just multiplication, but both more complicated and
simpler operations as well -- for example, addition, doubling,
squaring, and inverting are all required.  There should be interesting
tradeoffs based on which of these operations are directly accelerated
by hardware as opposed to which are mapped onto simpler units.

The control logic for an ECC processor is not simple: some necessary
algorithms take thousands of cycles\cite{mmm-hw_ecc} to compute even
when hardware accelerated.  Due to the length of these computations,
there should be some interesting tradeoffs based on how much of the
state machine gets put into hardware vs how much is driven by the
Rocket core.

\section{Benchmarking}

OpenSSL contains a fast, widely used, software implementation of
ECDSA\cite{kasper-openssl_ecc}.  This software implementation running
on an unmodified Rocket core will form our baseline to compare
against.

\bibliographystyle{plain}
\bibliography{bibliography}

\end{document}
