\documentclass[twocolumn]{article}
\usepackage{fullpage}
\usepackage{hyperref}

\title{CS250: An Elliptic Curve Cryptography Engine}
\author{Palmer Dabbelt and Kevin Linger}
\date{October 8, 2013}

\begin{document}
\maketitle

For our CS250 project we will be implementing an elliptic curve
cryptography (ECC) engine that attaches via the Rocket accelerator
interface.  Specifically we will be implementing an accelerator for
the elliptic curve digital signature algorithm (ECDSA), a NIST
standard widely used on the Internet for digital signatures. We will
build off of previous work by doing a more involved design-space
exploration. 

\section{Elliptic Curve Cryptography}

Digital signature algorithms\cite{fips-186-3} are a widely used class
of algorithms that are central to the functioning of the modern
Internet.  DSA\cite{us-dsa} (the original digital signature algorithm
as standardized by NIST) is a specialization of the ElGamal signature
scheme\cite{elgamal-sig} that uses discrete logarithms over finite
integer fields.  Finite integer fields are particularly easy to
compute with electronics, so this gives DSA the advantage of being an
efficient signature algorithm.

The problem is that the discrete logarithm problem is not particularly
strong over finite integer fields (specifically, there are known
sub-exponential algorithms\cite{adleman-subexp}).  As shown in
Table~\ref{key-sizes}, DSA does not scale well to higher security
levels.  This is important because code breaking algorithms are
getting more efficient, leading to the need for higher security
levels, therefore imposing higher computational costs on every node in
the system. 

\begin{table}[h]
  \begin{center}
    \begin{tabular}{cccc}
      Security Level & DSA & ECDSA & Ratio \\
      \hline
      80 & 1024 & 160 & 3:1 \\
      112 & 2048 & 224 & 6:1 \\
      128 & 3072 & 256 & 10:1 \\
      192 & 7680 & 384 & 32:1 \\
      256 & 15360 & 521 & 64:1 \\
    \end{tabular}
  \end{center}

  \caption{Comparison of DSA and ECDSA\cite{nsa-case_for_ecc}
    \label{key-sizes}}
\end{table}

The commonly accepted solution to this problem is to change the field
from integers to elliptic curves, which (as shown in
Table~\ref{key-sizes}) scale much better to higher security levels.
While elliptic curves are significantly more difficult to build
electronics for, this will eventually be offset by sheer key size
alone.  The currently accepted security level is 128 bits, at which
ECDSA already has a 10:1 advantage in key size. Shortening the key 
size becomes even more important,as we approach a world in which 
ubiquitous embedded systems must be able to securely transmit and 
receive data.  

\section{ECDSA in Hardware}

Unlike finite integer fields, elliptic curves do not map directly to
the hardware present in current microprocessors\cite{kss-ecdsa}. This
suggests that a hardware accelerator designed to compute over elliptic
curves should lead to a significantly more efficient implementation
than in software alone. ECDSA requires more computation per bit than DSA, 
so a hardware accelerator is necessary to fully take advantage of the 
smaller key size. 

Previous projects\cite{nnll-ecdsa_hw} have shown that ECC hardware can
be implemented efficiently, but were limited in the amount of
design-space exploration that was attempted. \cite{mmm-hw_ecc} explored 
the tradeoffs of different algorithms for computation, but did not explore 
the design space of any particular algorithm.  There
are three interesting axes on which to do design-space exploration
upon: the size of functional units, the types of functional units, and
the location of the control logic.

Despite the fact that ECC involves smaller keys, the numbers are still
quite large: 80 bits is the smallest viable key size, while 256 bits
is a reasonable but large key size.  As such, the multiplier can be built 
as either one large block  which consumes a lot of power, or be composed of 
smaller, chained multipliers that could be turned off to save power when a
small key size is used.

ECC demands not just multiplication, but both more complicated and
simpler operations as well -- for example, addition, doubling,
squaring, and inverting are all required.  There should be interesting
tradeoffs based on which of these operations are directly accelerated
by hardware as opposed to which are mapped onto simpler units.  For
example, we could choose to avoid building an inversion unit and
instead map it onto the existing multiplication units.

The control logic for an ECC processor is not simple: some necessary
algorithms take thousands of cycles\cite{mmm-hw_ecc} to compute even
when hardware accelerated.  Due to the length of these computations,
it will almost certainly be useful to integrate a state machine
directly into the accelerator so we can avoid burning 
energy driving the accelelator from the Rocket pipeline.  There should
be interesting tradeoffs between area, power, and programability
inside here.

\section{The Rocket Core Interface}

ECDSA has very light memory requirements: we simply load a ~256-bit
value from memory and then store the same size value out to memory,
while computing on it for thousands of cycles.  This should provide an
interesting space to explore: we can put memory hardware in the
accelerator at the cost of some area, or piggyback on the Rocket
core's memory hardware and push data over the coprocessor interface.

In addition to memory concerns, we are planning on investigating how
much control logic should live in the co-processor, and how much
should live on the Rocket core.  As all of the ECDSA operations that
would be sane to expose to the Rocket core are quite long (256-bit
curve-addition is probably the fastest, and it will be dozens of
cycles), the latency of the Rocket co-processor interface shouldn't be
too hard to deal with.

For ECDSA, the biggest limiting factor will be the low-width
connection to the Rocket core -- ECDSA has to deal with pretty large
numbers, and pushing them over a 64-bit pipe will probably
significantly limit the amount of design-space exploration we can do.
In other words, it would be wonderful to be able to get rid of storage
inside the accelerator to provide a very small co-processor, but it
will probably be too difficult to move data between the Rocket core
and the co-processor so that probably won't be an option.

The ECC will make limited use of the Rocket Core Interface. Our 
coprocessor will introudce two instructions: Sign and Validate. More
instructions may be introduced if the Rocket core becomes responsible
for handling part of the state machine. The instruction will also need
to be able to communicate the key length to the coprocessor. 

\section{Benchmarking}

OpenSSL contains a fast, widely used, software implementation of
ECDSA\cite{kasper-openssl_ecc}.  This software implementation running
on an unmodified Rocket core will form our baseline to compare
against. For analysis of power and area, we can compare to \cite{mmm-hw_ecc} 
and \cite{nnll-ecdsa_hw}. 

\section{Timeline}

The design space of ECDSA naturally lends itself to a progression.
Essentially the plan is to start with an entirely software-based
implementation, and then accelerate specific functions until we end up
with a more full-fledged accelerator.  We will hopefully be able
to paramaterize the design sufficiently that we can keep the smaller
cores around for a while in order to run data. We will explore the design-
space as we progress.

\begin{itemize}
\item October 8: Project Proposal
\item October 29: Software ECDSA running on Rocket
\item November 19: Adder and Multiplier running on co-processor
\item December 7: Inverter running on co-processor
\end{itemize}

\bibliographystyle{plain}
\bibliography{bibliography}

\end{document}
