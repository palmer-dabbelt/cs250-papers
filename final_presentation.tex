\documentclass{beamer}
\usepackage{graphicx}
\usepackage{multicol}
\usepackage{vwcol}
\usepackage{bm}

\title{Elliptic Curve Digital Signature Algorithm}
\author{Kevin Linger and Palmer Dabbelt}
\date{December 13, 2013}

\begin{document}
\maketitle

\begin{frame}
  \frametitle{Public Key Cryptography}


  \begin{itemize}
  \item Two keys (called a keypair)
    \begin{itemize}
    \item Public: the whole world can know this
    \item Private: only the owner knows this
    \end{itemize}
  \item Four operations
    \begin{itemize}
    \item \texttt{Sign(Message, Private)} $\rightarrow$ \texttt{Signature}
    \item \texttt{Verify(Message, Signature, Public}) $\rightarrow$ \texttt{Boolean}
    \item \texttt{Encrypt(Message, Public)} $\rightarrow$ \texttt{Cyphertext}
    \item \texttt{Decrypt(Cyphertext, Private}) $\rightarrow$ \texttt{Message}
    \end{itemize}
  \item ECDSA is a widely-used (SSH, SSL/TLS) signature algorithm
    \begin{itemize}
    \item Supports \texttt{Sign} and \texttt{Verify}
    \end{itemize}
  \end{itemize}
\end{frame}

\begin{frame}
  \frametitle{Cryptography and the Swarm}

  \begin{itemize}
  \item Original motivation: the universal dataplane
    \begin{itemize}
      \item Large network ($10^{10}$ nodes)
      \item Many low power sensors ($\bm{\mu} \mathbf{W}$)
      \item Some high power servers
    \end{itemize}
  \item Goal: all storage is fault-tolerant
    \begin{itemize}
    \item Byzantine fault tolerence algorithm
    \item Sign every message on the sensors
    \end{itemize}
  \item Problem: ECDSA is $\mathbf{mW}$ per signature!
  \end{itemize}
\end{frame}

\begin{frame}
  \frametitle{The Discrete Logarithm Problem}

  \begin{itemize}
  \item $b^k = g$
    \begin{itemize}
    \item Difficult in one direction (given $b$ and $g$ find $k$)
    \item Easy in another direction (given $b$ and $k$ find $g$)
    \end{itemize}
  \item Only asymetric for some groups
    \begin{itemize}
    \item $({\mathbb Z}_p)^\times$: Integers modulo some prime,
      repeating multiplication
    \item $GF(p)$: Elliptic curves modulo some prime, repeating
      addition
    \end{itemize}
  \item At the center of many public-key crypto schemes
  \end{itemize}
\end{frame}

\begin{frame}
  \frametitle{Integers}

  \begin{itemize}
  \item 
  \end{itemize}
\end{frame}

\begin{frame}
  \frametitle{Elliptic Curves}

  Discuss the benifits and drawbacks of ECDSA vs DSA
  \begin{itemize}
  \item Key generation: $Q_A = d_A * G$
    \begin{itemize}
    \item Given $d_A$ and $G$, $Q_A$ is easy to find
    \item Given $Q_A$ and $G$, $d_A$ is hard to find
    \end{itemize}
  \end{itemize}
\end{frame}

\begin{frame}
  \frametitle{ECDSA}

  \begin{center}
    \begin{multicols}{2}
      \includegraphics[width=0.9\linewidth,height=0.7\textheight,keepaspectratio]{ecdsa_sign.uncrop.pdf} \\
      \includegraphics[width=0.9\linewidth,height=0.7\textheight,keepaspectratio]{ecdsa_verify.uncrop.pdf} \\
    \end{multicols}
  \end{center}
\end{frame}

\begin{frame}
  \frametitle{Point Multiplication}

  \begin{itemize}
  \item ECDSA is essentially point multiply and some cleanup
    \begin{itemize}
    \item Point Multiply: O(million) cycles
    \item Cleanup: O(thousand) cycles
    \end{itemize}
  \end{itemize}
\end{frame}

\begin{frame}
  \frametitle{Modular Multiply}

  Show that modmul is just a bunch of integer operations.
\end{frame}

\begin{frame}
  \frametitle{Software Implementation(s)}

  OpenSSL, my implementation
\end{frame}

\begin{frame}
  \frametitle{Hardware Implementation}

  Discuss paramaterizability
\end{frame}

\begin{frame}
  \frametitle{Comparison}

  x86/openssl, rocket/mine, rocc/point add, rocc/point mul
\end{frame}

\begin{frame}
  \frametitle{Future Work}

  Affine representation, montgomery multipliers, parallel software,
  ...
\end{frame}

\end{document}
